\documentclass{article}

\oddsidemargin = -0.25in
\headheight = 12pt
\textheight = 9in
\hoffset = 0pt
\topmargin = -0.5in
\headsep = 25pt
\textwidth = 7in

\title{A minimal equipment one-pot RT-LAMP assay for detecting reverse transcriptase activity}

\begin{document}
\maketitle
%\tableofcontents
%\newpage

\section{Introduction}

There are a number of chronic diseases including cancers, neurological diseases, autoimmune diseases, and immune deficiencies with a suspected retroviral etiology \cite{voisset2008human}. In addition, tests for known retroviruses with a high viral diversity such as HIV are very specific and could miss divergent HIV strains \cite{bartolo2012hiv}\cite{luft2011hiv}. Patients looking for an explanation of their chronic disease with a suspected retroviral etiology face significant challenges in diagnosis due to political impositions. While assays for detecting reverse transcriptase activity are available to researchers, the US government makes it illegal for patients to access these assays as the tests have not been approved for clinical diagnosis. While a reverse transcriptase activity assay by itself cannot diagnose a specific retrovirus, it can provide supporting evidence to look further with more specific investigations.

% access to researcher, commercial options equipment
A patient who would like to investigate the presense of reverse transcriptase activity must either find a researcher who would be willing to run the test, travel to a country whose government allows such testing, perform social engineering to have a sample processed by pretending to be a researcher, or purchase the necessary equipment and supplies and run tests themselves. While all avenues should be explored to maximize the chances of success, the goal here is to improve access by developing a relatively inexpensive test that can be performed at home with minimal equipment requirements. The limitation of this approach is that in countries such as the US, the government, insurance companies, and doctors largely stand in the way of the patient acting on this information. However, at the very least, it is additional evidence to improve the confidence of the patient to pursue a retroviral cause through whatever means possible.

\section{Existing Reverse Transcriptase Assays}



\section{Design}

design constraints balance minimal design/development time, minimal equipment requirements, adequate performance.

\subsection{Sample Collection}

minimal sample volume, blood drop from lanclet 

\subsection{Lysis}

\cite{malmsten2005reverse}

\cite{karamohamed1998bioluminometric}

\cite{curtis2008rapid}

\subsection{Reaction}



\bibliographystyle{ieeetr}
\bibliography{rt_activity}


\end{document}
