\documentclass{article}

\oddsidemargin = -0.25in
\headheight = 12pt
\textheight = 9in
\hoffset = 0pt
\topmargin = -0.5in
\headsep = 25pt
\textwidth = 7in

\title{A minimal equipment one-pot RT-LAMP assay for detecting reverse transcriptase activity}

\begin{document}
\maketitle
%\tableofcontents
\newpage

\section{Introduction}

There are a number of chronic diseases including cancers, neurological diseases, autoimmune diseases, and immune deficiencies with a suspected retroviral etiology \cite{voisset2008human}. In addition, tests for known retroviruses with a high viral diversity such as HIV are very specific and could miss divergent HIV strains \cite{bartolo2012hiv}\cite{luft2011hiv}. Patients looking for an explanation of their chronic disease with a suspected retroviral etiology face significant challenges in diagnosis due to political impositions. While assays for detecting reverse transcriptase activity are available to researchers, the US government makes it illegal for patients to access these assays as the tests have not been approved for clinical diagnosis. While a reverse transcriptase activity assay by itself cannot diagnose a specific retrovirus, it can provide evidence to look further with more specific investigations.



\section{Design}



\subsection{Sample Collection}

minimal sample volume, blood drop from lanclet 

\subsection{Lysis}

\cite{malmsten2005reverse}

\cite{karamohamed1998bioluminometric}

\cite{curtis2008rapid}

\subsection{Reaction}



\bibliographystyle{ieeetr}
\bibliography{rt_activity}


\end{document}
